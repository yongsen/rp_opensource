%\documentclass[conference]{IEEEtran}
\documentclass[draftclsnofoot,journal,onecolumn,12pt]{IEEEtran}

\usepackage{graphicx}
\usepackage{bm}
\usepackage[bookmarks=true,pdfstartview=FitH]{hyperref}
\usepackage{bookmark}
\usepackage{algpseudocode}
\usepackage{algorithm}
\usepackage[caption=false]{subfig}
\usepackage{url}
\usepackage{threeparttable}

% correct bad hyphenation here
\hyphenation{op-tical net-works semi-conduc-tor}

\begin{document}

\title{Open Source Software Development Process}

\author{\IEEEauthorblockN{Yongsen MA} \\
\IEEEauthorblockA{Shanghai Jiao Tong University \\
E-mail: mayongsen@gmail.com}
}

% make the title area
\maketitle
%
%\begin{abstract}
%%\boldmath
%The abstract goes here.
%\end{abstract}


\section{Statement of Purpose}
I want a detailed description of the process. For example, how are crash reports handled? How are bug reports handled? How are bugs classified and confirmed? How are the assignments to individual developers made? How to merge code changes in Git? How are code inconsistency handled? In each step of the process, have you identified any software engineering issues which have rooms for improvements?

\href{http://www.ted.com/talks/aaron_koblin.html}{Aaron Koblin: Artfully visualizing our humanity}

\href{http://www.mpt.net.nz/2012/06/why-free-software-has-poor-usability/}{Why free software has poor usability, and how to improve it}

\section{Introduction}

\subsection{motivating, joining, participating and contributing}

\begin{enumerate}
  \item acquire: knowledge, experience, opportunities; backup, platform
  \item participate: happiness, communication
  \item contribute: freedom, trustworthy
\end{enumerate}

\begin{enumerate}
  \item developer
  \item user(evaluation)
\end{enumerate}

\begin{enumerate}
  \item public
  \item private
\end{enumerate}

\subsection{modeling, examination, investigation}

\begin{enumerate}
  \item individuals
  \item groups
  \item organizations
\end{enumerate}

\begin{enumerate}
  \item operate systems
  \item web
  \item application
  \item network
\end{enumerate}

\begin{enumerate}
  \item contribute:
  \item process: stable, scalable
  \item acquire: software, individuals, groups
\end{enumerate}

\begin{enumerate}
  \item graph theory
  \item multiproject
  \item interdependent
\end{enumerate}

\section{Maintaining}

How are the assignments to individual developers made? How to merge code changes in Git? How are code inconsistency handled? In each step of the process, have you identified any software engineering issues which have rooms for improvements?

\subsection{Components}

\begin{enumerate}
  \item Home Page
  \item Code Repository
  \item Mailing List
  \item Bug Tracking System
  \item Wiki
\end{enumerate}

\subsection{Participate}

\begin{enumerate}
  \item Starting
  \item Discussion
  \begin{itemize}
    \item Subscribe Mailing List
    \item Take part in News Group
    \item Participate in Conference
  \end{itemize}
  \item Programming
  \begin{itemize}
    \item Consummate documents
    \item Running test codes
    \item Report Bugs
    \item Submit patch
  \end{itemize}
  \item Improving
\end{enumerate}

\subsection{Git and ath9k}


\section{Debugging}

For example, how are crash reports handled? How are bug reports handled? How are bugs classified and confirmed?

\subsection{Basic Debugging}

\subsection{Functional Debugging}

\section{Improvement}

\nocite{*}

\renewcommand\refname{References}
\bibliographystyle{abbrv}
%\IEEEtriggeratref{6}
\bibliography{open}
%%\printbibliography

\end{document}
